In this paper we show how the categorization of solar wind can be informed by classic unsupervised clustering methods and Self-Organizing Maps (SOM). We demonstrate that a single technique used in isolation can be misleading for the interpretation of automatic classifications. We show that it is important to examine the SOM lattices, in conjunction with the solar wind fingerprints and the time series. Thanks to these tools we can differentiate classes associated with known heliospheric events.

We are convinced that basic unsupervised clustering techniques will have difficulties in finding characteristic solar wind classes when they are applied to unprocessed data. A combination of feature engineering, non-linear autoencoding and SOM training leads to a more appropriate segmentation of the data points.

The classification of the solar wind also depends on the objectives that want to be attained: if the goal is to classify the solar wind to study its origin on the Sun, features related to solar activity must be included in the model; however, if the goal is to identify geoeffectiveness, other parameters should be added to the list of features, including geomagnetic indices.

All the tools and the techniques presented here can be applied to any other data set consisting of large amounts of points with a fixed number of properties. SOMs have already been used in astrophysics\citep{Suveges2017} and magnetospheric physics\citep{Camporeale2018}, and we are currently working on their deployment for the study of active regions on the Sun.

All the software and the data used in this work are freely available for reproduction and improvement of the results presented above.

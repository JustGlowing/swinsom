\subsubsection{Data Set Used}
The solar wind data used in this work was obtained by the Advanced Composition Explorer (ACE) spacecraft, during a period of 14 years, between 1998 and 2011. The data can be downloaded from the \href{ftp://mussel.srl.caltech.edu/pub/ace/level2/multi}{FTP servers of The ACE Science Center (ASC)}. The files in this repository correspond to a compilation of hourly average data from three instruments: MAG (Magnetometer), SWEPAM (Solar Wind Electron, Proton, and Alpha Monitor) and EPAM (Electron, Proton, and Alpha Monitor). A detailed description of the entries in this data set can be found in the \href{http://www.srl.caltech.edu/cgi-bin/dib/rundibviewmultil2/ACE/ASC/DATA/level2/multi}{ASC website} listed in section \ref{sec:repos}.

A total of 122712 data points are available. However, routine maintenance operations, low statistics, instrument saturation and instrument degradation produce gaps and errors in the data. The SWICS data includes a flag assessing the quality of the calculated plasma moments. We retain only `Good quality' entries. Fig.\ref{fig:datacoverage} shows the final number of items retained per year. Our pre-processed data set contains a total of 72454 points.

\subsubsection{Additional Derived Features}
We created additional features for each entry, based on expert and previous knowledge of the physical properties of the solar wind. These new features are derived from the existing properties in the data set or computed from statistical analysis of their temporal evolution.

Multiple techniques have been proposed in the literature to identify ejecta, Interplanetary Coronal Mass Ejections (ICME), and solar wind origins in the ACE data. \citep{Zhao2009} suggest that during solar cycle 23 three classes of solar wind can be identified using the solar wind speed, $V_{sw}$, and the oxygen ion charge state ratio, $O^{7+}/O^{6+}$. The classification boundaries based on these three parameters are presented in Table \ref{tab:swtypes}. This solar wind classification will be known in this manuscript as the Z classification.
cme
\citep{Xu2015} suggested an alternative four class classification based on the proton-specific entropy, $S_p = T_p/n_p^{2/3}$, the Alfv\'en speed, $V_A = B / (\mu_0 m_p n_p)^{1/2}$, and the velocity-dependent expected proton temperature, $T_\text{exp} = (V_{sw}/258)^{3.113}$. The classification conditions based on these three parameters are also presented in Table \ref{tab:swtypes}. This solar wind classification will be known in this manuscript as the X classification. For each entry in the data set we have included the values of $S_p$, $V_A$, $T_\text{exp}$, and the solar wind type given by the two classification methods. Additional auxiliary variables, like the Alfv\'en Mach number ($M_A$) and the temperature ratio ($T_\text{exp}/T_p$), have also been included in the data set.

In addition to these instantaneous quantities, we can perform statistical operations over a window of time of six hours, including values of the maximum, minimum, mean, standard deviation, variance, auto-correlation, and range. These quantities can be used to take into account temporal variations of the solar wind over the previous few hours.

Two additional terms, which have been successfully used in the study of solar wind turbulence \citep{SEE ROBERTS REFS}, are included here to account for additional time correlations: the normalized cross-helicity, $\sigma_c$ eq. \eqref{eq:sigmac}, and the normalized residual energy, $\sigma_r$ eq. \eqref{eq:sigmar}, where $\boldsymbol{b} = \left(\boldsymbol{B}- \boldsymbol{\left<B\right>}\right)/(\mu_0m_pn_p)^{1/2}$ is the fluctuating magnetic field in Alfv\'en units, $\boldsymbol{v} = \boldsymbol{V_{sw}}- \boldsymbol{\left<V_{sw}\right>}$ is the fluctuating solar wind velocity, $\boldsymbol{z^\pm} = \boldsymbol{v} \pm \boldsymbol{b}$ are the Els\"asser variables \citep{Elssaser1950, Magyar2019}, and $\left<.\right>$ denotes the averaging of quantities over the time window.

\begin{align}
\sigma_c & = 2 \left< \boldsymbol{b}\cdot\boldsymbol{v}\right>/\left<\boldsymbol{b}^2 + \boldsymbol{v}^2\right> \label{eq:sigmac} \\
\sigma_r & = 2 \left< \boldsymbol{z^+}\cdot\boldsymbol{z^-}\right>/\left<\boldsymbol{z^-}^2 + \boldsymbol{z^+}^2\right> \label{eq:sigmar}
\end{align}

Due to gaps in the data, some of the above quantities can not be calculated. We eliminate from the data set all entries for which the derived features presented in this section could not be calculated. This leaves a total of 69672 entries in the data set used in the present work.

To account for the differences in units and scale, each feature column $\boldsymbol{F}$ in the data set is normalized to values between 0 and 1, using: $\boldsymbol{f}=\left(\boldsymbol{F}-\min{\boldsymbol{F}}\right) /\left(\max{\boldsymbol{F}}-\min{\boldsymbol{F}}\right)$.

\subsubsection{Complementary Data Catalogs}
We support the interpretation of our results using data from three solar wind event catalogs. The first is the well known Cane and Richardson catalog that contains information about ICMEs detected in the solar wind, ahead of the Earth \citep{Cane2003} \citep{Richardson2010} \footnote{\href{http://www.srl.caltech.edu/ACE/ASC/DATA/level3/icmetable2.htm}{Near-Earth Interplanetary Coronal Mass Ejections Since January 1996: http://www.srl.caltech.edu/ACE/ASC/DATA/level3/icmetable2.htm}}. We used the August 16, 2019 revision of this catalog. As the authors state in their website, there is no spreadsheet or text version of this catalog and offline editing was necessary. We downloaded and re-formatted the catalog to use it in our application. The CSV file created has been made available in our repository. We call this, the Richardson and Cane catalog. 

The second catalog correspond to the ACE List of Disturbances and Transients \footnote{\href{http://www.ssg.sr.unh.edu/mag/ace/ACElists/obs\_list.html}{ACE Lists of Disturbances and Transients: http://www.ssg.sr.unh.edu/mag/ace/ACElists/obs\_list.html}} produced by the University of New Hampshire. As in the previous case, the catalog is only available as an html webpage, so we have manually edited the file and extracted the catalog data into a CSV file also available in our repository. This is here referred as the UNH catalog.


Finally, we also included data from the Shock Database \footnote{\href{https://www.cfa.harvard.edu/shocks/ac_master_data/}{Harvard-Smithsonian, Center for Astrophysics, Interplanetary Shock Database - ACE: https://www.cfa.harvard.edu/shocks/ac\_master\_data/}} maintained by Dr. Michael L. Stevens and Professor Justin C. Kasper at the Harvard-Smithsonian Center for Astrophysics. Once again we have gathered and edited multiple web-pages in a single CSV file available in our repository. In this work this will be known as the CfA catalog.
\subsubsection{Model Amaya-21}

\paragraph{General overview}

Fig.\ref{fig:maps} (A) confirms that the code words are well distributed in the latent space. The feature and components maps show in addition that lattice nodes share common attributes with their neighbors. The regularity in the colors of the feature map confirms that, even when the components depict a complex pattern, the SOM keeps its most important feature: organization. This is clear by the proximity of neighboring code words in the latent space, marked with red lines. We expect then to find common patterns in all the following maps in this section.

In the same figure, black and white lines indicate relative distances between the lattice nodes: thicker lines represent larger inter-node distances. This shows that there are sections of the map (groups of code words in the latent space) that can form separate groups. This group separation is more evident in the `Class map' shown in Fig.\ref{fig:compmap}: the $k$-means clustering of the code words divides the space following the inter-nodal divisions. The effects of this classification can also be observed in the third row of Fig.\ref{fig:clustering}. This scatter plot of all the points is colored by the SOM classes, both in the auto-encoder space (panels e and f) and in the PCA transformed space (panels k and l). The class distributions is more complex than the one obtained with the $k$-means and the GMM techniques.

Back in Fig.\ref{fig:maps} the `Hit map' presents a good distribution of points among all the lattice nodes, except in regions that show isolation from the rest. This represent solar ind types that have atypical properties. One of the goals of the DSOM is to cover those isolated zones where rare events can be classified in separate nodes. In contrast, the classic SOM method tends to cluster the code words in regions of high density, troubling the categorization of rare events, like ICMEs, ejecta or magnetic clouds, in the solar wind data.

Fig.\ref{fig:datarange} shows the distribution of the SOM training data in the normalized range $\left[0..1\right]$. This is a violin plot superposed by a box plot, showing the data distribution for each one of the features listed in table \ref{tab:features}. Normalization of the data is performed using the maximum and minimum values of each feature, but outliers (extremely large or small values) can hinder the use of particular features. In any classification problem, outlier detection and elimination is extremely important. The figure shows that all our data points are well represented in the data. Even in some cases, like for feature 13 (Alfv\'en Mach number), where the distribution has a small width, it still covers a significant part of the total range. This figure also shows that feature 6 (cross-helicity, $\sigma_c$) has a bimodal distribution, with two peaks close to the limits. This shows a significant part of the data lays close to the $\sigma_c = \pm 1$ limit. The cross-helicity is a measure of the Alfv\'enicity of the solar wind, representing the direction and intensity of the propataion of Alfv\'enic fluctuations. At the Earth orbit this an indication of the origins of the solar wind in the north or south hemispheres of the Sun.

Feature 20 (Pearson auto-correlation of the magnetic field magnitude) also shows a large distribution function, with a marked peak near one. This quantity was calculated for a window of time of six hours and a time lag of one hour. It shows the extent to which the values of the magnetic field have changed in one hour. High autocorrelation values represent situations in which the magnetic field does not change during the window, i.e. values in the window at time $t$ are the same as values shifted by one hour, $t-1$. Completely uncorrelated signals, produced by random changes in time, will produce autocorrelation values close to 0 (0.5 after normalization), i.e. the data in the window at time $t$ is different from the data in the shifted window. Positive (negative) values represent a signal with a periodicity of one hour. Additional time lags could be used to create extra features, but here we use only one to test its effectiveness.

\paragraph{Class and Feature Maps}

Lattice nodes are characterized by their weights (code). Applying a reverse transformation, followed by a re-scaling, we obtain their values in the original N-dimensional space. Fig.\ref{fig:compmap} shows the DSOM clustering and the 21 feature values associated to each lattice node. We have clustered the nodes in eight classes. This is a subjective selection inspired by other works in the literature. In our case the clustering leads to contiguous node groups, except for class zero (0) and class three (3), composed of two groups of nodes. This division can have multiple origins, but we can not discard the possibility of a suboptimal DSOM convergence.

However, the maps show the properties that differentiate each of the classes. In the case of class three (3) we observe that it corresponds to zones where the range of the proton density and magnetic field component are larger than the rest. This class is also composed of high average iron charge, iron to oxygen ratios, ionized oxygen ratios, and compressed wind, expressed by the large density and magnetic field magnitudes. All this points to a class related to magnetic clouds.

Class zero (0) contains solar wind with negative cross-helicity, small magnetic field range, and low density range. This is evidence of the importance of the cross-helicity: the large bimodal distribution observed in Fig.\ref{fig:datarange} is strong enough to separate the data points and force distinct classes. The cross-helicity map shows a clear separation of nodes in positive and negative values to the right and to the left of the map. This division is evident in the final clustering observed in the `Class map'. Class zero can be associated to uncompressed wind flowing in the direction opposite to the IMF. This is probably associated with north hemispheric coronal hole origins, as the dominant polarity on the solar surface during solar cyle 23 was negative in the north pole. This is also based on the fact that the highest speeds observed in the solar wind velocity map are located in nodes that make part of class zero.

Another remarkable class is number five (5). This class presents a higher isolation from its neighbors and presents a very low number of hits, as shown in Fig.\ref{fig:maps}. This indicates that it contains solar wind rarely observed. It is characterized by high levels of iron ionization, and high iron to oxygen, and oxygen 7 to oxygen 6, ratios. The class also presents low proton-specific entropy ($S_p$), low proton temperature, high temperature ratio, low proton density and very high Alfv\'en velocities. This class is fully composed by events that have high one hour autocorrelations. All these characteristics point to short explosive events with high magnetic field component. These are characteristics of magnetic clouds. The low residual energy, $\sigma_r$, and the neutral values of the cross-helicity, $\sigma_c$, seem to point in the same direction.

\paragraph{Solar Wind Fingerprints}

SOMs show the variability of solar wind and how it can be visually characterized. The SOM is a helpful guide in the study of the different types of solar wind, but is not necessarily an objective, unbiased and final classification method. SOM opens the possibility for a fast visual characterization of large and complex data sets. Here we have performed a quick informed interpretation of 15 years of very complex data.

The characterization of the different SOM classes, performed in the previous section, can also be summarized in Fig.\ref{fig:classesdatarange}. This figure is a mix of figures \ref{fig:datarange} and \ref{fig:compmap}: each row corresponds to a single solar wind class, represented by its 21 features using box plots. The colors correspond to the class number (0 to 7 from top to bottom). The first column has been built from a classification using $k$-means, the second with GMM and the third with DSOM. Here we can find again the properties described in the previous section for classes zero, three and five. We call these plots class `fingerprints'.

We also see that different classification methods lead to different classes with different fingerprints. But visual inspection of the fingerprints is more difficult to interpret than the SOMs. \citep{Roberts2020} and \citep{Xu2015b} performed detailed descriptions of particular solar wind classes based on the mean values observed in each subset of points. But Fig.\ref{fig:classesdatarange} shows that some features can have a very large distribution. For example the values of the solar wind speed, feature 2, has a very large spread on all the classes and all classifications, except for class 3, 4, 5 in the $k$-means classification, and classes 1, 3, 4 and 5 of the SOM classification. Other features seem to have a large fingerprint spread, including features 6 (cross-helicity) and 7 (residual energy), but these are produced by the bimodal nature of the features, as shown in the violin plots of \ref{fig:datarange}.

In our analysis of the solar wind classification in the previous section, we took into account the level of dispersion of each feature in the class fingerprint. We also noticed that in the $k$-means classification multiple classes have almost the same fingerprint, at the exception of a single feature. For example classes 0, 1 and 2 have almost the same characteristics except for features 6 (cross-helicity) and 20 (magnetic field autocorrelation). The same is true for classes 2, 3 and 6 in the GMM classification. SOM classes tend to present fingerprints that are more variable. This is a consequence of the use of the DSOM version of the SOM that does not agglomerate nodes in zones with high density of points. On the contrary, $k$-means and GMM will tend to put more points in high density zones, creating very similar class fingerprints.

\paragraph{Time series}
A more classical analysis of the solar wind is performed by experts using visual inspection of the properties of the solar wind during time windows. Fig.\ref{fig:timeseries} presents a window of time of four months, from the beginning of May 2003 to the end of September 2003. We have plotted entries in the Richardson and Cane, UNH and CfA catalogs on top of time series of the solar wind speed in panels a, b and c. These plots have been colored by the class number of the $k$-means, GMM and SOM classes.

The polarity of the solar wind can be observed in panel d. Changes from red to green are associated with crossings of the heliospheric current sheet. The z-component of the magnetic field is plotted in blue (positive) and red (negative) in panel e.

Panel f shows the evolution of the $O^{7+}/O^{6+}$ ratio using a dotted black line in logarithmic scale. The Z classification boundaries (see table \ref{tab:swtypes}) are plotted using a black continuous line. The red area corresponds to `non-coronal hole origin' solar wind, and data points receive this classification if the dotted line enters the red zone. If it stays above it, the solar wind is considered an ICME. If the curve drifts bellow the red zone, the wind is considered to have origins in a coronal-hole.

First we see how our characterization of the SOM classes performed in the previous sections is expressed in this plot. Our analysis suggests that class 3 of the SOM correspond to magnetic clouds. In the time series of panel three we clearly see a correlation between the catalog entries for shocks and ICMEs and the class 3. This is particularly clear in the data at the end of the month of May, during one of the ICMEs of mid-June, in the shock crossings of end June, and partially during the last four ICME entries from mid-July until end of August. Panel d seems to point to boundary structures in the sector reversal zone from north to south solar hemisphere. However, Fig.\ref{fig:tsfeatures-som} shows that this class is mainly associated with very strong oxygen ion ratios, high iron average charge and strong magnetic fields, associated with ejecta, and already identified in the previous section by the SOMs.

Class 0 was expected to be solar wind with north solar hemisphere origins. This is confirmed by the plots in Fig.\ref{fig:tsfeatures-som}, where the cross-helicity is close to -1. This figure also shows that classes 0 and 2 are closely related. Their fingerprints only differ by their magnetic field autocorrelation. This seem to be almost the same type of solar wind.

Class 5 was expected to belong to magnetic clouds from the analysis of the maps, but presents very low numbers in the hit map. In the time series, it is only visible on zones close to shock entries in the UHN and CfA catalogs, around the events of end-May and mid-August. This entries are more visible in the iron to oxygen plot in Fig.\ref{fig:tsfeatures-som}. It is also interesting to see that in mid-July an event was also identified as class 5. It has no entry in any of the catalogs and was not singled out by any other of the clustering techniques. This was observed already by \citep{Roberts2020}. The DSOM is able to pick such an outlier thanks to a better coverage of the latent space.

The time series also shows that SOM class 4 contains data with near zero cross-helicity, hight average iron charge (`avqFe'), above average iron to orygen ratios and a high oxygen ion ratio. This was also found by \citep{Roberts2020} and is believed to be correlated with ICMEs connected in both ends to the Sun, with waves propagating in both directions along the field lines. This is also visible in the maps of Fig.\ref{fig:compmap}, but is more difficult to characterize due to variability in the class properties, caused by the low number of hits in the corresponding lattice nodes.

Classes 6 and 7 of the SOM classification have been grouped together in the previous section. In the time series we observe that these classes share a high cross-helicity, and bellow average iron charge. Their difference is visible in the value of the magnetic field auto-correlation (not shown in the time series).

This analysis of the time series confirms the suspicion risen in the map analysis section. It also confirms some of the observations by other authors during the exact window of time. A full interpretation of the automatic classification is then possible combining maps, fingerprints and time series verification.

\paragraph{Maps of Empirical Classifications}
SOM allows visual analysis of previously published results. In this section we show how the Xu and Zhao classifications activate different nodes of the SOM. We use two properties of the SOM simultaneously: the size of the lattice nodes will represent the number of hits for a particular class, and the color will represent one property of the solar wind.

We take 3 subsets of the full data set, corresponding to the points categorized as CHW, ICME and NCHW respectively in the Zaho catalog. Each one of these three subsets is passed through the Amaya-21 model and we observe how each one activates the SOMs. All properties are normalized between zero and one, using the maximum and minimum values for each feature in the full data set.

Fig.\ref{fig:SWtypeZ} shows the SOMs of the three Zhao classes produced by the Amaya-21 model. The maps can be interpreted in the following way: CHW, ICME and NCHW classes have different number of hits. These solar wind types activate different nodes in the lattice. Each row in the figure shows a different activation of the map for each one of the three subsets. We see that the color for each class is almost the same in each map but is different between the classes. This means that the oxygen ion state ratio and the solar wind speed can be used to separate the three different classes as proposed by \citep{Zhao2009}.

[CHECK??] We observe in particular that the ICME class is mainly contained in the zone corresponding to class 5, which has been previously identified as such in the map and time series analysis sections above. Here we see that the total number of hits is only 155, which explains why it is so difficult to observe in the time series. This is an additional benefit of using SOMs: we are able to detect important data points that can easily be overlooked with other methods.

In a similar way, Fig.\ref{fig:SWtXu} shows the SOMs of the Xu classification. This time we used four subsets of the data set, each one corresponding to a different Xu class. 

[CHECK??] Once again `ejecta' is confined to the region of SOM class 5, and `sector reversal origin' solar wind activates nodes also activated by the non-coronal hole wind in the Z classification. This same zone is also overlapping with `streamer belt origin' zones in the X classification. This class seems to be included in the SOM class 2, and in part in class 3. Finally, `coronal hole origin' solar wind is confined mainly to class 2 and to the corner node of the map. This corner is characterized by a very low oxygen and iron ionization, very high speed and temperature, and very low density. It is possible to isolate this unique node and study more in detail all its characteristics, but this is out of the scope of the present work and will be presented in a future pubication.

\subsubsection{Model Roberts-8}
The same techniques used for model Amaya-21 were applied this model. The only difference between these two models is the amount of data used and the selected initial features. We can see that these two modifications can have an important impact in the analysis. One of the most important differences is the long density wings observed in the density plot of Fig.\ref{fig:modelR}. This is a projection in the latent space after transformation using an autoencoder. There is an evident bimodal distribution in this space, mainly due to the importance of the cross-helicity term that divides all the solar wind data. The composition of the features is slightly more difficult to observe here, as the lower left zone, that contains the least amount of hits, shares very similar first and third components. In this particular model most of the points have been mapped to lattice nodes on the three other corners of the map.

The time series in Fig.\ref{fig:modelR} shows that the model can differentiate zones of high and low speed, as well as zones of polarity inversion and some of the shocks and ICMEs. But it struggles to separate more meaningful characteristics. These results are not completely compatible with those of \citep{Roberts2020}, even for the $k$-means classification plotted in panel a. This shows that a re-run of the $k$-means classification with the same data and the same number of classes can lead to different results. Without a complete analysis using SOMs and fingerprints it is impossible to converge to a singular objective fully automatic classification.
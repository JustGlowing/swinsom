The effects of solar activity on the magnetic environment of the Earth have been observed since the publication of Edward Sabine's work in 1852 \citep{Sabine1852}. During almost two hundred years we have learned about the intimate connection between our star and the plasma environment of the Earth. Three main physical processes connect the Sun to us: the transfer of electromagnetic radiation, the transport of energetic particles, and the flow of solar wind. The later is a continuous stream of charged particles that carries the solar magnetic field out of the corona and into the interplanetary space.

The name `solar wind' was coined by Parker in 1958 because `the gross dynamical properties of the outward streaming gas [from the Sun] are hydrodynamic in character'\citep{Parker1958}. Over time we have learned that the wind also has many more complex properties. Initially, it was natural to classify the solar wind by defining a boundary between `fast' and `slow' winds \citep{Habbal1997}. The former has been associated with mean speed values of 750 km/s, while the later shows a limit at 500 km/s, where the compositional ratio (Fe/O) shows a break \citep{Feldman2005,Stakhiv2015}. The solar wind also carries information about its origins on the Sun. At certain solar distances the ion composition of the solar wind is expected to be frozen-in, reflecting the electron temperature in the corona and its region of origin \citep{Feldman2005,Zhao2009,Stakhiv2015}. These particles have multiple energies and show a variety of kinetic properties, including non-Maxwellian velocity distributions \citep{Pierrard2010,Matteini2012}.

The solar wind is also connected to the Sun by the Interplanetary Magnetic Field (IMF), thorough magnetic field lines directed towards the Sun, away from the Sun, or in the case of flux ropes connected at both ends \citep{Owens2016,Gosling2010}. The thin region where solar magnetic fields of opposite directions meet is called the Heliospherc Current Sheet (HCS). When a spacecraft crosses the HCS instruments onboard can detect the change in magnetic field direction as a 180$^\circ$ reversal. Changes in the flow properties are also observed around the HCS. This perturbed zone is called the Heliospheric Plasma Sheet (HPS), and the passage of the spacecraft from one side of the HPS to the other is known as a Sector Boundary Crossing (SBC) \citep{Winterhalter1994}. In spacecraft observations these are sometimes confused with Corotating Interaction Regions (CIR), which are zones of the solar wind where fast flows have caught up with slow downstream solar wind, compressing the plasma.

From the point of view of a spacecraft SBCs and CIRs can show similar sudden changes in the plasma properties. These two in turn are often grouped and mixed with other transient events, like Coronal Mass Ejections (CME) and Magnetic Clouds (MC). Since 1981 when \citep{Burlaga1981} described the propagation of MC behind an interplanetary shock, it was suspected that CMEs and MC where coupled. However, more recent studies show that CMEs observed near the Sun do not necessarily become MC, but instead `pressure pulses' \citep{Gopalswamy1998,Wu2006}.

Much more recently it has been revealed, by observations from Parker Solar Probe, that the properties of the solar wind can be drastically different closer to the Sun, where the plasma flow is more pristine and has not yet mixed with the interplanetary environment. Patches of large intermittent magnetic field reversals, associated with jets of plasma and enhanced Poynting flux, have been observed and named `switchbacks' \citep{Bale2019,Bandyopadhyay2020}.

The solar wind is thus not only an hydrodynamic flow, but a compressible mix of different populations of charged particles and electromagnetic fields that carry information of their solar origin (helmet streamer, coronal holes, filaments, solar active regions, etc.) and is the dominion of complex plasma interactions (ICMEs, MC, CIRs, SBCs, switchbacks).

To identify and study each one of these phenomena we have relied in the past on a manual search, identification and classification of spacecraft data. Multiple authors have created empirical methods of wind type identification based on in-situ satellite observations and remote imaging of the solar corona. Over the years the number and types of solar wind classes has changed, following our understanding of the complexity of heliospheric physics.

Solar wind classification serves three main roles:
\begin{enumerate}
	\item it is used for the characterization of its origins in the corona,
	\item to identify the conditions where the solar wind is geoeffective,
	\item to isolate different plasma populations in order to perform statistical analysis.
\end{enumerate}

Among these classifications we can include the original review work by \citep{Withbroe1986}, the impressive continuous inventory by \citep{Richardson2000,Richardson2010,Richardson2012}, and the detailed studies by \citep{Zhao2009} and \citep{Xu2015b}. These publications classify the solar wind based on its origins and on the transient events detected. Each system includes two, three or four classes, generally involving coronal-hole origins, CMEs, streamer belt origins and sector reversal regions.

We are moving now towards a new era of data analysis, where manual human intervention can be replaced by `intelligent' software. The trend has already started, with the work by \citep{Camporeale2017b} who used the \citep{Xu2015b} classes to train a Gaussian Process algorithm that autonomously assigns the solar wind to the proper class, and more recently by \citep{Roberts2020} who used unsupervised classification to perform a 4 and 8 class solar wind classification. Machine learning algorithms have been used in the past in other applications in solar physics \citep{Lundstedt1996,Qahwaji2008c,Ahmed2013c,Bobra2015,Bobra2016,Nishizuka2017c,Camporeale2018}, and astrophysics \citep{VanderPlas2012,Ntampaka2015,Hajian2015,Suveges2017,Bai2018,Bonjean2019}.

The most basic machine learning techniques learn using two methods: a) in supervised learning the algorithms are shown a group of inputs and ouputs with the goal to find a non-linear relationship between them, b) in unsupervised learning the machine is presented with a cloud of multi-dimensional points that have to be autonomously categorized in different classes. This means that we can program the computer to learn about the different types of solar wind using the existing empirical classifications, or by allowing it to independently detect patterns in the solar wind properties.

In the present work we show how the second method, unsupervised classification, can be used to segregate different types of solar wind. In addition, we show hot to visualize and interpret such results. The goal of this paper is to introduce the method to the community, the best use practices and the opportunities that such system can bring. We implement one specific type of classification, called Self-Organizing Maps, and we compare it to simpler classification techniques, showing that it can reveal hidden information in years of solar wind data.

In the next sections we present in detail the techniques of data processing (section \ref{sec:dataprocessing}), data dimension reduction (sections \ref{sec:reducpca} and \ref{sec:reducae}) and data clustering (section \ref{sec:clustering}) that we have used. We then present in detail the Self-Organizing Map technique and all its properties in section \ref{sec:som}. We show how to connect all of these parts together in section \ref{sec:fullarchi}, and finally we show how the full system can be used to study 14 years of solar wind data from the ACE spacecraft in section \ref{sec:results}.


%\citep{Gosling2010} : flux ropes, subset of CME, lasting ~12hours, low beta, smoothyl rotating.
%\citep{Wu2006} : MC duration of at least 8h
 
%/!\ for the tieme series figure: the HCS is generally easily identified by an approximately 180° change in direction of the magnetic field spiral angle over one to a few hours [Zhou et al., 2005]
%
%
%Magnetic reconnection solar wind-magnetosphere 1985
%
%
%Prediction model of solar wind geo-effectiveness using bayersian theory base on solar wind features 1997
%
%Geo-effectiveness of MCs, IPSs, and CIRs. 2011. CIRs alone in 2006
%
%Solar sources of geoeffective structures. Webb 2001. 
%
%Origins of the solar wind in the corona. Withbroe. Two clear sources: CH and CME. Role of streamers? role of things between? Role of spicules and polar plumes?
%
%Habbal, 1997: slow solar wind is associated with the axes, also known as stalks, of streamers. Furthermore, the ultraviolet observations also show how the fast solar wind is ubiquitous in the inner corona and that a velocity shear between the fast and slow solar wind develops along the streamer stalks
%
%Origin on the Sun of magnetic clouds and their structure: CME are flux ropes originating in filaments in the sun. 1986.
%
%** Solar wind properties and their effect on Earth. REVIEW of controversial and uncontroversial Solar sources: fast sw is from CH. Fast streams are known to be connected to the central regions of large coronal holes. Slow streams, however, appear to come from a wide range of sources, including streamers, pseudostreamers, coronal loops, active regions, and coronal hole boundaries. Forecasting techniques review. Cranmer 2017.
%--> Also: controversy about how to properly classify the sw.
%--> Also: solar wind features fade in time/distance --> Prepare for PSP!
%
%More detailed-recent classification using ACE: Coronal Sources and In Situ Properties of the Solar Winds Sampled by ACE During 1999 – 2008. Hui Fu 2015.
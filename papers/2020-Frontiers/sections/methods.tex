\subsubsection{Dimension Reduction using PCA}
\label{sec:reducpca}
Principal Component Analysis (PCA) is a mathematical tool used in data analysis to simplify and extract the most relevant features in a complex data set. This technique is used to create entries composed of linearly independent `principal components'. These are the eigenvectors of the covariance matrix $\Sigma$ applied to the centered data, eq.\eqref{eq:covariance}, ordered from the largest to the smallest eigenvalue, $\lambda_1 \ge \lambda_2 \ge ... \ge \lambda_n$, where $\overline{\boldsymbol{X}}$ is the mean value of each original feature, eq.\eqref{eq:xmean}. The projection of the data onto the principal component space ensures a maximal variance on the direction of the first component. Each subsequent principal component is orthogonal to the previous ones and points in the direction of maximal variance in the residual sub-space \citep{Shlens2014}.

\begin{align}
\overline{\boldsymbol{X}} & = \frac{1}{m} \sum_{i=1}^{m} \boldsymbol{X}_i \label{eq:xmean} \\
\Sigma & = \frac{1}{m} \sum_{i=1}^{m} \left( \boldsymbol{X}_i - \overline{\boldsymbol{X}} \right)\left( \boldsymbol{X}_i - \overline{\boldsymbol{X}} \right) \label{eq:covariance}
\end{align}

The PCA transformation creates as many components in the transformed space, $\boldsymbol{\tilde{X}}$, as features in the original data space $\boldsymbol{X}$. However, components with small eigenvalues belong to a dimension where the variance is so small that it is impossible to separate points in the data. It is a general practice in data reduction to keep only the first $k$ components that explain at least a significant portion of the total variance of the data, $\lambda_{i=1..k}/\text{Tr}(\Sigma) > \epsilon$. This allows for a selection of information that will effectively differentiate data points, and for a reduction of the amount of data to process during analysis. Many techniques have been suggested for the selection of the values of $k$ and the cut-off $\epsilon$ \citep{Rea2016}. We use the value of $k=3$ to simplify the comparison among the different models, but for a detailed study of the solar wind, if a PCA transformation is applied, it is important to use a fixed criteria for the selection of the cut-offs.

Fig. \ref{fig:dimreduc} (A) is a 3D scatter plot of all the data points, colored by the Xu classification, projected on the first three PCA components. The features used to create this figure are presented in section \ref{sec:fourmodels}. Panel (B) contains the same data colored by the Zhao classification. These projections show that the Xu and Zhao classification are defined by hyper-planes separating the points, even if the data has been linearly transformed by the PCA. Class 2, ICMEs-ejecta, is restricted to a small domain in this coordinate system (for both the Xu and Zhao classification). The lateral plots on panel (A) are 2D histograms of the point distribution on the three main PCA planes. They show that the concentration of points is not homogeneous and different zones can be isolated using unsupervised classification techniques. There is a clear segregation of points in the (1st,2nd)-component plane: as we will see in subsequent sections, one of the features of the solar wind presents a strong bimodal distribution that is prioritized by the PCA.

\subsubsection{Dimension Reduction Using Autoencoders}
\label{sec:reducae}
PCA has a limitation: the principal components are a linear combination of the original properties of the solar wind. An alternative to data reduction is the use of autoencoders (AE). These are machine learning techniques that can create non-linear combinations of the original features projected on a latent space with less dimensions \citep{Hinton2006}. This is accomplished by creating a system where an encoding function, $\phi$, maps the original data $\boldsymbol{X}$ to a latent space, $\boldsymbol{\mathcal{F}}$, eq.\eqref{eq:encoder}. A decoder function, $\psi$, then maps the latent space back to the original input space, eq.\eqref{eq:decoder}. The objective of the autoencoder is to minimize the error between the original data and the data produced by the compression-decompression procedure as shown in eq.\eqref{eq:aeminimization}.

\begin{align}
\phi: & \boldsymbol{X} \rightarrow \boldsymbol{\mathcal{F}} \label{eq:encoder}\\
\psi: & \boldsymbol{\mathcal{F}} \rightarrow \boldsymbol{X} \label{eq:decoder} \\
\phi,\psi = & \underset{\phi,\psi}{\arg \min} \left\lVert \boldsymbol{X} - (\phi \circ \psi) \boldsymbol{X} \right\rVert^2 \label{eq:aeminimization}
\end{align}

Autoencoders can be represented as feed-forward neural networks, where fully connected layers lead to a central bottleneck layer with few nodes and then expands to reach again the input layer size. An encoded element, $\boldsymbol{z} \in \boldsymbol{\mathcal{F}}$, can be obtained from a data entry, $\boldsymbol{x} \in \boldsymbol{X}$, following the standard neural network function, eq.\eqref{eq:encodex}, where $\boldsymbol{W}$ is the weights matrix, $\boldsymbol{b}$ is the bias, and $\sigma$ is the non-linear activation function.

\begin{align}
\boldsymbol{z} & = \sigma \left( \boldsymbol{W}\boldsymbol{x} + \boldsymbol{b} \right) \label{eq:encodex} \\
\boldsymbol{\hat{x}} & = \sigma' \left( \boldsymbol{W'}\boldsymbol{z} + \boldsymbol{b'} \right) \label{eq:decodez} \\ 
\mathcal{L}(\boldsymbol{x}, \boldsymbol{\hat{x}}) & =  \left\lVert \boldsymbol{x}- \boldsymbol{\hat{x}} \right\rVert^2 \label{eq:aeloss}
\end{align}

The decoding procedure, shown in eq.\eqref{eq:decodez}, transforms $\boldsymbol{z}\rightarrow\boldsymbol{\hat{x}}$, where the prime quantities are associated with the decoder. The loss function, $\mathcal{L}(\boldsymbol{x}, \boldsymbol{\hat{x}})$, is the objective to be minimized by the training of the neural network using gradient descent. Once training is completed, the vector $\boldsymbol{z}$ is a projection of the input vector $\boldsymbol{x}$ onto the lower dimensional space $\boldsymbol{\mathcal{F}}$.

Additional enhancements and variations of this simple autoencoder setup exist in the literature, including multiple regularization techniques to minimize overfitting \citep{7407967}, Variational Autoencoders (VAE) that produce encoded Gaussian distribution functions \citep{Kingma2013}, and Generative Adversarial Networks (GAN) that produce new (unseen) data \citep{Goodfellow2014}. In this work we use the most basic form of autoencoders, presented above.

The second column of Fig.\ref{fig:dimreduc}, panels (C) and (
D), contains the same information as the first column, but with the data set encoded in the three dimensional latent space $\boldsymbol{\mathcal{F}}$. Panel (C) shows that all the classes in the Xu and Zhao classification are easy to distinguish, including ICMEs-ejecta (class 2) that is difficult to discern in the PCA. This projection also shows that class 4 from the Zhao classification in the bottom panels, overlaps with class 3 (sector reversal origin), and partially with class 2 (ejecta) in the Xu classification on the top panels. Panel (C) shows, on the side planes, 2D histograms of the density of points. These can be seen as the volume integral of the point density in each direction. Here again it is possible to observe multiple zones of high concentration, suggesting that multiple types of solar wind are present in the data and that they can be differentiated using an unsupervised classification technique. 

\subsubsection{Clustering Techniques}
\label{sec:clustering}
The goal of unsupervised machine learning is to group data points in a limited number of clusters in the N-dimensional space $\Omega\in\Bbb R^N$, where N is the number of features (components or properties) in the data set. Multiple techniques can be used to perform multi-dimensional clustering. We present in Fig. \ref{fig:clustering} the three clustering techniques used to classify our 3D reduced data. The panels in the first column show the data projected in the PCA reduced space, $\boldsymbol{\tilde{X}}$, while the second column shows the data in the latent AE space, $\boldsymbol{\mathcal{F}}$. Each row corresponds to a different clustering method. The colors in the top panels (A) and (D) were obtained using the $k$-means method \citep{1056489}, the colors in the middle panels (B) and (E) were obtained using the Gaussian Mixture Model (GMM) \citep{bishop2006machine}. The bottom panels are colored by the classes from the Self-Organizing Maps described later in section \ref{sec:som}.

The $k$-means technique has already been used in a recent publication for the determination of solar wind states \citep{Roberts2020}. To our knowledge other clustering methods have never been used in the literature to classify the solar wind, but \citep{Dupuis2020} has used the GMM to characterize magnetic reconnection regions in simulations using their velocity distribution information.

The colors used in Fig.\ref{fig:clustering} are assigned randomly by each clustering technique. The most glaring issue with them is that different methods can lead to different clusters of points. The GMM and the $k$-means agree on their classification in the PCA space, but show dissimilar results in the AE space. Moreover, for a single method, e.g. $k$-means, slight modifications of the clustering parameters, e.g. using a different seed for the random number generator, can lead to very different results. We address this last issue using an algorithm that launches the $k$-means and GMM algorithms 500 times until the methods converge to a quasi-steady set of clusters. But we warn that the results are implementation dependent.

In the present data set, the cloud of points is convex and well distributed in all three components. This raises one additional issue, observed more clearly in the first column of Fig.\ref{fig:clustering}: when classical clustering methods are applied to relatively homogeneously dense data, it divides the feature space in Vorono\"i regions with linear hyper-plane boundaries. This is an issue with all clustering techniques based on discrimination of groups using their relative distances (to a centroid or to the mean of the distribution). To avoid this problem density-based techniques, such as DBSCAN \citep{ester1996density}, and agglomeration clustering methods, use a different approach. However, we can not apply them here because in such homogeneous cloud of points these techniques lead to a trivial solution where all data points are assigned to a single class.

There is no guarantee that a single classification method, with a particular set of parameters will converge to a physically meaningful classification of the data if the points in the data do not have some level of separability, or have multiple zones of high density. This is also true for other classification methods based on `supervised learning'. In those applications same issues will be observed if the training data uses target classes derived from dense data clouds using simple hyper-plane boundaries, as done for the Zhao and Xu classes. An example of such application was published by \citep{Camporeale2017b}. The authors used the Xu classification to train a Gaussian Process classifier.

\subsubsection{Self-Organizing Maps}
\label{sec:som}

\paragraph{Classical SOM}

Following the definitions and notations by \citep{Villmann2006}, a class can be defined as $C_i\overset{\text{def}}{=} \{x\in\Omega | \Phi(x)=\boldsymbol{w}_i\}$, where $\Phi$ is a function from $\Omega$ to a finite subset of $k$ points $\{\boldsymbol{w}_i\in\Bbb R^N\}_{i=1..k}$. A cluster $C_i$ is then a partition of $\Omega$, and $\{\boldsymbol{w}_i\}$ are the code words (also known as nodes, weights or centroids) associated. The mapping from the data space to the code word set, $\Phi: \Omega\rightarrow\mathcal{W}$, is obtained by finding the closest neighbor between the points $\boldsymbol{x}$ and the code words $\boldsymbol{w}$, eq.\eqref{eq:winner}. The code word $\boldsymbol{w_s}$, the closest node to the input $\boldsymbol{x_s}$, is called the `winning element'. The class $C_i$ corresponds to a Vorono\"i region of $\Omega$ with center in $\boldsymbol{w}_i$.

\begin{equation}
\Phi: x \rightarrow  \underset{i\in\mathcal{N}}{\arg\min}\left( \left\lVert x - \boldsymbol{w}_i \right\rVert \right) \label{eq:winner}
\end{equation}

A Self-Organizing Map (SOM) also composed of structured nodes arranged in a lattice, and assigned to a fixed position $\boldsymbol{p}_i$ in $\Bbb R^q$, where $q$ is the dimension of the lattice (generally $q=2$). The map nodes are characterized by their associated code words. The SOM learns by adjusting the code words $\boldsymbol{w}_i$ as input data $x$ is presented.

The SOM is the ensemble of code words and nodes $A_i = \{ \boldsymbol{w}_i, \boldsymbol{p}_i\} \in (\Omega \times \Bbb R^q)$. For a particular entry $\boldsymbol{x_s}$, the code word $s \in \mathcal{N}$ is associated to the winning node $\boldsymbol{p_s}$ if the closest word to $\boldsymbol{x_s}$ is $\boldsymbol{w_s}$. At every iteration of the method, all code words of the SOM are shifted towards $x$ following the rule:

\begin{equation}
\Delta \boldsymbol{w}_i = \epsilon(t)h_\sigma(t,i,s)(x-\boldsymbol{w}_i) \label{eq:learnsom}
\end{equation}

with $h_\sigma(t,i,j)$ defined as the lattice neighbor function:

\begin{equation}
h_\sigma(t,i,j) = e^{-\frac{\left\lVert \boldsymbol{p}_i - \boldsymbol{p}_j \right\rVert^2}{2\sigma^2(t)}} \label{eq:neigsom}
\end{equation}

where $\epsilon(t)$ is the time dependent learning rate, eq.\eqref{eq:epsilon}, and $\sigma(t)$ is the time dependent lattice neighbor width, eq.\eqref{eq:sigma}. The training of the SOM is an iterative process where each data point in the data set is presented to the algorithm multiple times $t={0, 1,..,t_f}$. In these equations the subscript $0$ refers to initial values at $t=0$ and the subscript $f$ to values at $t=t_f$.

\begin{align}
\epsilon(t) & = \epsilon_0 \left(\frac{\epsilon_f}{\epsilon_0}\right)^{t/t_f} \label{eq:epsilon} \\
\sigma(t) & = \sigma_0 \left(\frac{\sigma_f}{\sigma_0}\right)^{t/t_f} \label{eq:sigma}
\end{align}

This procedure places the code words in the data space $\Omega$ in such a way that neighboring nodes in the lattice are also neighbors in the data space. The lattice can be presented as a $q$-dimensional image, called map, where nodes sharing similar properties are organized in close proximity.

The main metric for the evaluation of the performance of the SOM is called the quantization error:

\begin{equation}
Q_E = \frac{1}{M} \sum_{i=1}^M \left\lVert x_i - w_{x_i} \right\rVert
\end{equation}

where $M$, is the total number of entries in the data set.

Once the training of the SOM is finished, the code words $\boldsymbol{w}_i$ can be grouped together using any clustering technique, e.g. k-means. The nodes of the SOM with close properties will be made part of the same class. The classes thus created are an ensemble of Vorono\"i subspaces, allowing a complex non-linear partitioning of the data space $\Omega$.

The final number of clusters is an input of the algorithm, but can also be calculated autonomously. The Within Cluster Sum of Squares (WCSS) can be used as a metric of the compactness of the clustered nodes. As its name implies the WCSS is the sum of the squred distances from each node to their cluster point. If only one class is selected, the large spread of the nodes would produce a high WCSS. The lowest possible value of the WCSS is obtained for a very high number of classes, when the number of classes is equal to the number of nodes. But such extreme solution is also unpractical. The optimal number of clusters can be obtained using the Kneedle class number determination \citep{5961514}. In the present work we do not use this technique: we will perform comparisons with previous publications that propose a fixed number of solar wind types. We will explore the use of an automatic class number selection in a future publication.

\paragraph{Dynamic SOM}
The time dependence of the SOM training allows the code words $\boldsymbol{w}_i$ to reach steady coordinates by slowing down their movement over the iterations. Due to the minimization of the distance in eq.\eqref{eq:winner} code words tend to agglomerate around high density zones of the feature space. The Dynamic Self-Organizing Map (DSOM), introduced by \citep{Rougier2011}, eliminate the time dependence and allows to cover larger zones of the space outside of the high density regions.

The DSOM is a variation of the SOM where the learning function \eqref{eq:learnsom} and the neighbor function \eqref{eq:neigsom} are replaced by eqs. \eqref{eq:learndsom} and \eqref{eq:neigdsom} respectively:

\begin{align}
\Delta \boldsymbol{w}_i & = \epsilon \left\lVert x - \boldsymbol{w}_i \right\rVert_\Omega h_\eta(i,s,x)(x-\boldsymbol{w}_i) \label{eq:learndsom} \\
h_\eta(i,s,x) & = e^{-\frac{1}{\eta^2}\frac{\left\lVert \boldsymbol{p}_i - \boldsymbol{p}_j \right\rVert^2}{\left\lVert x - \boldsymbol{w}_s \right\rVert_\Omega^2}} \label{eq:neigdsom} 
\end{align}

where $\epsilon$ is a constant learning rate, $h_\eta(i,s,x)$ is defined as the new lattice neighbor function, and $\eta$ is the `elasticity' parameter. In their work \citep{Rougier2011} show that DSOM can be used to better sample the feature space $\Omega$, reducing the agglomeration of code words around high density zones. The DSOM does not converge to a steady solution, due to the lack of a temporal damping factor.

\paragraph{Visualization of SOM and DSOM}
Clustering techniques do not necessarily convergence to a steady immutable solution. Differences in the training parameters or slight changes in the data can have an important impact on the final classification. These tools can be used for statistical analysis, comparisons, data visualization and training of supervised methods. But it will be practically impossible to claim the existence of a general objective set of states discovered only by the use of these basic clustering techniques.

However, SOMs and DSOMs provide an important tool for the study of the solar wind: the maps are composed of nodes that share similar properties with its immediate neighbors. This allows for visual identification of patterns and targeted statistical analysis.

Fig.\ref{fig:maps} shows the basic types of plots and maps that can be generated using the SOM/DSOM techniques. This figure uses data from the model Amaya-21 which has been encoded, using AE, into a set of entries, $\boldsymbol{z}_i$, each one composed of three components. Panel (A) shows a histogram of the first two components of the feature space $\Omega$, with dots marking the position of the code words $\boldsymbol{w}_i$. The colors of the dots represent their SOM classification. The red lines connect a single code word $\boldsymbol{w}_s$ with its six closest neighbors. The panel (B) shows the `hit map' of the SOM. It contains the lattice nodes $\boldsymbol{p}_i$ associated to the code words $\boldsymbol{w}_i$. They are depicted as hexagons with sizes representing the number of data points connected to each node and colored by their SOM class. The thickness of the lines between lattice nodes represent the relative distance to its neighbors in the feature space $\Omega$. Red lines connect the node $\boldsymbol{p}_s$, associated to the code word $\boldsymbol{w}_s$ in panel (A), to its closest neighbors.

Panel (C) of Fig.\ref{fig:maps} corresponds to the value of a single feature associated to each node; as an example we use the ionized oxygen ratio $O^{7+}/O^{7+}$ (`O7to6'). To improve visualization all hexagon sizes have been set to their maximum and the inter-node distance line has been colored white. In order to obtain the correct value for each node, we must first perform a decoding of the data from the latent space, $\Omega=\boldsymbol{\mathcal{F}}$, to the original data set space, $\boldsymbol{X}$.

Panel (D) of Fig.\ref{fig:maps} shows that the nodes of the lattice can also be used to present data that has not been used in the training of the SOM. The method keeps track of the points in the data set associated to each lattice node, it is then possible to perform independent statistical operations on those points alone. Moreover, it is possible to activate the SOMs with just a subset of the data, i.e. with points that feature a specific solar wind type. In this case, as an example, we have colored the map using the average oxygen charge state $\left<Q_{O}\right>$ (`avqO'), and we have set the size of the nodes to represent the frequency of points with solar wind type Xu=2 (ejecta). The dark line between the lattice nodes designate the boundaries between different SOM classes.

These four representations are only a few examples of the variety of data that can be represented using SOMs. The most important aspect of the SOMs is that data is represented in simple 2D lattices where the nodes share properties with their neighbors. Here we also decided to use hexagonal nodes, connecting 6 equidistant nodes, but other types of representations are also valid, e.g. squared or triangular nodes.

The bottom row of Fig.\ref{fig:maps} displays all three components of the code words $\boldsymbol{w}_i$ associated with each one of the $\boldsymbol{p}_i$ nodes. In the first panel they have been mapped to the basic colors Red, Green and Blue (RGB). The remaining panels have been colored using each individual component. The first panel is then the RGB composition of the three remaining ones where the boundaries between the SOM classes have been highlighted.

\subsubsection{The Full Architecture}
\label{sec:fullarchi}
The previous sections introduced all the individual pieces that we use for the present work. Here we give a global view of the full model. Fig.\ref{fig:architecture} shows how all the components are interconnected. The data set is composed of clean and processed entries. We tested the PCA transformation in cases Amaya-21 and Roberts-8, keeping only the first three principal components. This possible setup is presented on the left of the figure. It is also possible to perform an unsupervised clustering directly on the un-processed data, as shown on the top of the figure, but it is not recommended. For the remaining of this manuscript we present only the cases where the non-linear AE encoding, shown at the right of Fig.\ref{fig:architecture}. The bottleneck of the AE network is three nodes, i.e. the data is encoded in three components. The transformed data is then used to train the SOM.

After training, the code words of the SOM are then clustered to group together nodes that share similar properties. This second level classification is done using the $k$-means++ algorithm with 500 re-initializations (it is in general recommended to use between 100 and 1000 iterations). The total number of classes selected is an input of the model and has been set to 8. This arbitrary choice was made following the results presented by \citep{Roberts2020}. All the software was implemented in Python using as main libraries PyTroch, Scikit-learn and Pandas.

\paragraph{Autoencoder architecture}
We use a basic, fully connected feed-forward neural network for the encoding-decoding process. The bottleneck of the network has been fixed to three neurons in order to simplify the visualization. This arbitrary choice is another parameter of the models that need further investigation. The neural network is symmetric in size but the weights of the encoder, $\boldsymbol{W}$, and the decoder, $\boldsymbol{W'}$, are not synchronized (see eqs.\eqref{eq:encodex}, \eqref{eq:decodez}). We use multiple fully connected hidden layers, where the central layer is the size of the bottleneck. Each layer is composed of a linear regressor, followed by batch normalization and a ReLU activation function. The output layer of the network contains a linear regressor followed by a hyperbolic tangent activation function. The autoencoder has been coded in python using the PyTorch framework.

We use an Adam optimizer \citep{Kingma2014} for the gradient descent with a learning rate of 0.001 and a weight decay of 0.0001 for regularization. The loss function is the Mean Squared Error (MSE). We train the network for 30 epochs, after which we see no additional improvement in the loss function. The full data set was randomly divided 50\%/50\% between training and testing sets.

\paragraph{Two Models of Solar Wind Classification}
\label{sec:fourmodels}
We have tested the two models presented in Table \ref{tab:features}. The models are inspired by the work of \citep{Roberts2020}. We call these cases Amaya-21 and Roberts-8.  The table lists all the features used in each model. A detailed description of each feature can be found in the \href{http://www.srl.caltech.edu/cgi-bin/dib/rundibviewmultil2/ACE/ASC/DATA/level2/multi}{ACE Level 2 documentation}. To spread the data over a larger range of values in each component, we have used the logarithm of all the quantities, except of those marked with an asterisk in the table.

Features 15 to 20 contain an additional suffix, corresponding to a statistical operation performed on the corresponding feature. The operations include the mean, the range, the standard deviation and the auto-correlation of quantities over a window of time of 6 hours. This window allows to capture temporal (spatial) fluctuations in some of the solar wind parameters.

On the lower part of Table \ref{tab:features} we present the range of dates used for each model. For Amaya-21 we use the full data set, while for the model Roberts-8 we try to replicate as much as possible the choices made in \citep{Roberts2020}. The same table also contains the hyper-parameters selected to run the two models. The number of neurons per layer in the encoding half of the neural network is listed in the table and was manually selected to minimize the final loss value of the AE.

All the figures presented until now correspond to the processing of data from model Amaya-21. The amount of data and figures produced in this work is very large and is not possible to include all of them in the present document. We will present in the next section some highlights, but more detailed analysis of each one of the cases will be presented in future publications.

\paragraph{Budget}
Machine learning models require fine tuning of different parameters, from the selection and testing of multiple methods, to the parameterization of the final architecture. \citep{Dodge2019} suggests that every publication in machine learning should include a section on the budget used for the development and training of the method. The budget is the amount of resources used in the data processing, the selection of the model hyper-parameters (HP), and its training.

The most time-consuming task in the present work has been the data preparation, the model setup and debugging and the writing of the SOM visualization routines. All the techniques described in the previous sections have been coded in python and are freely accessible in the repositories listed in section \ref{sec:repos}. We estimate the effort to bring this work from scratch to a total of 2 persons month. Of these, one person week was dedicated to the manual testing an selection of different model HPs (autoencoder architecture, feature selection, learning rates, initialization methods, number of epochs for training, selection of data compression method, size of the time windows, etc.).

All classic clustering techniques presented in section \ref{sec:clustering} require only a few lines of code and can be trained in minutes on a mid-range workstation (e.g. Dell Precission T5600, featuring two Intel(R) Xeon(R) CPU E5-2643 0 @ 3.30GHz with four cores and eight threads each). The most time consuming tasks of our models are the training of the autoencoder (5\% od the total run time), the multiple passages of the clustering algorithms (15\% of the run time), and the optimization of the SOM hyper-parameters (80\% of the run time). The training of the SOM is performed in less than a minute.

For reference, the total run-time for each one of the models used in this work are: 60 minutes for the Amaya-21 model and 20 minutes for the Roberts-8 model.

\paragraph{Hyper-Parameter Optimization}
Our main goal in this manuscript is to introduce the use of the SOMs for the classification of solar wind data. SOMs require the selection of four main Hyper-Parameters (HPs): the size of the lattice, $(m\times n)$, the initial learning rate, $\epsilon_0$, and the initial neighbor radius, $\sigma_0$. In the case of the DSOM algorithm, these two last HPs are replaced by the constant learning rate, $\epsilon$, and the elasticity, $\eta$. The automatic selection of the best HP for machine learning model is called Hyper-Parameter Optimization (HPO).

We use the library `optuna' \citep{Akiba2019} to perform an automatic optimization of the four HPs. The optimization is based on a technique called Tree-structured Parzen Estimator (TPE) \citep{pmlr-v28-bergstra13}. The objective function of the optimizer was set to minimize the expression:

\begin{equation}
\mathcal{H}\left(\sigma, \eta, m, n\right) = Q_E(\sigma, \eta, m, n) + \alpha \frac{m}{m_{max}} + \beta \frac{n}{n_{max}} + \gamma m n \label{eq:hpo}
\end{equation}

where $Q_E$ is the SOM quantization error, $m$ and $n$ are the number of lattice nodes in each dimension, and $m_{max}$ and $n_{max}$ are the given maximum number of possible nodes. The weight factors $\alpha$, $\beta$ and $\gamma$ are used to impose more or less restictions on each term. This optimization objective allows to minimize the quantization error, but also the number of nodes in the SOM. The squaring term $\gamma m n$ forces the map to be as squared as possible.

After a total of 100 trial runs of the model using different HPs, the optimizer selected the parameters presented in the lower section of table \ref{tab:features}. The total run time for the HPO of the case Amaya-21 is about 40 minutes. HPO is, understandably, one of the most expensive procedures in all our setup.